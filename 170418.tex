\subsection{Cheney-Scan}
Stop-the-world semispace copy-and-collect garbage collector that we will be
implementing. \\

\subsubsection{Heap Traversal}
white = unvisited \\
black = visited \\
gray = partially visited \\

Initially, gray the root. Then, color all of the nodes reachable from the root
also gray, and once we do, color the root black. \\
Gray nodes are ones that are copied to the to space. \\
Black nodes are ones whose members are also copied to the to space. \\
Be careful when you need to gray a node that is already grayed. (eg: just
update the pointer etc.) \\
Done with collection when have nothing but black and white nodes. \\
Note:
Cheney-Scan happens to be BFS, but DFS may actually provide better locality. \\

Tri-color Invariant:
\begin{itemize}
  \item no edges from black to white
  \item any reachable white object is reachable from a gray object
\end{itemize}

\subsubsection{Implementation}
AP ``Allocation Pointer'' \\
LP ``Limit Pointer'' \\

When mutator asks for n words, increment AP by (n + 1), first one being the
H ``Header'', and then the n allocated words after it. \\
Before the incrementing, check if ap + (n + 1) $>$ lp, and if so, gc. \\

On gc, set AP and LP to the ends of the to-space. \\
Then, copy over roots. \\
Then, copy over nodes connected to those roots. \\
Then, copy over nodes connected to those nodes, etc. \\
Define gray nodes to be all nodes between TP and AP, where TP ``Trace Pointer''
is maintained appropriately, and is incremented as we black away a node. \\
Define black nodes to be all nodes between start of to space and TP. \\


\section{Elaboration}
\begin{enumerate}
  \item \sout{Syntatic} Semantic Sugar
  \item Nonsense
    \begin{itemize}
      \item open
      \item include
    \end{itemize}

  % things that make sense, but we don't have a good representation for them yet
  \item no \underline{good} type theory yet
    \begin{itemize}
      \item avoidance problem
      \item pattern matching
    \end{itemize}
\end{enumerate}

\subsection{Target Language}
\begin{flalign*}
  exp &\bnfdef &\\
  dec &\bnfdef &\\
  ty  &\bnfdef &\\
  pat &\bnfdef &\\
  match &\bnfdef &\\
  mod &\bnfdef &\\ % contain decs
  sig &\bnfdef &\\
  spec &\bnfdef &\\ % in between mods and sigs
  id  &\bnfdef &\\
  % sml doesn't have variables, it only have identifiers (can be variables,
  % but can also be module names etc)
\end{flalign*}

Identifier Resolution is an important topic (where did the `x' come from).

\subsection{Judgement}
% This gamma is the il-module context
% Mec and \sigmaec are also IL-module pieces
% the exp is EL
% e and tau are also IL-Module pieces
If $\Gamma \vd M_{ec} \of \sigma_{ec} \rtri exp \leadsto e \of \tau$
and $\vd \Gamma \ok$, $\Gamma \vd M_{ec} \of \sigma_{ec}$,
then $\Gamma \vd e \of \tau$ \\
$\Gamma \vd M_{ec} \of \sigma_{ec} \rtri id \leadsto e \of M \of \sigma$ \\

Also, extend IL-Module with names (and certain ``namespaces''). \\
\begin{flalign*}
  \sigma &\bnfdef \dots \bnfalt \name \of \sigma &\\
  \name  &\bnfdef \texttt{VAL } id \bnfalt \texttt{CON } id
          \bnfalt \texttt{MOD } id \bnfalt \texttt{FUN } id \bnfalt \dots &\\
\end{flalign*}

eg:
$M \of \Sigma\bind{\alpha \of \sigma_1}{\VAL{foo} \of \sigma_2} \rtri \VAL{foo}
  \leadsto \outn{\pi_2 M}$
