\documentclass[11pt]{article}

%\usepackage{amsmath}
\usepackage{amssymb}
\usepackage{mathtools}
\usepackage{latexsym}
\usepackage{proof}
\usepackage[margin=0.5in]{geometry}
\usepackage{mathpartir}
\usepackage{graphicx} % for the nice downarrow
\usepackage{xifthen}
\usepackage{cancel}
\usepackage{tikz}
\usepackage[normalem]{ulem}

\usepackage{titlesec}
\titleformat*{\section}{\large\bfseries}
\titleformat*{\subsection}{\bfseries}
\titleformat*{\subsubsection}{}
%\renewcommand{\thesetion}{\arabic{section}}
%\renewcommand{\thesubsection}{(\alpha{subsection})}


\usepackage{listings}
\usepackage{color}
\definecolor{eclipseBlue}{RGB}{42,0.0,255}
\definecolor{eclipseGreen}{RGB}{63,127,95}
\lstset {
  basicstyle=\small\ttfamily,
  captionpos=b,
  tabsize=2,
  columns=fixed,
  breaklines=true,
  mathescape=true,
% frame=l,
% numbers=left,
% numberstyle=\small\ttfamily,
  morekeywords= {
    EQUAL, GREATER, LESS, NONE, SOME, abstraction, abstype, and, andalso, array, as, before, bool, case, char, datatype, do, else, end, eqtype, exception, exn, false, fn, fun, functor, handle, if, in, include, infix, infixr, int, let, list, local, nil, nonfix, not, o, of, op, open, option, orelse, overload, print, raise, real, rec, ref, sharing, sig, signature, string, struct, structure, substring, then, true, type, unit, val, vector, where, while, with, withtype, word
  },
  morestring=[b]",
  morecomment=[s]{(*}{*)},
  stringstyle=\color{black},
  identifierstyle=\color{eclipseBlue},
  keywordstyle=\color{red},
  commentstyle=\color{eclipseGreen}
}

\setlength\parindent{0pt}

% samepage doesn't quite work in some cases, not sure why
% TODO: 16pt is an approximation
\newenvironment{grouped}[1]{\begin{minipage}{\textwidth}#1}{\end{minipage}\vspace{16pt}}

\newcommand{\trans}[1]{$\xrightarrow{\textnormal{#1}}$}
\newcommand{\tab}{\hspace*{2pt}}
\newcommand{\arrow}{\mathbin{\rightarrow}}
\newcommand{\bnfdef}{\mathrel{\Coloneqq}}
\newcommand{\bnfalt}{\mathrel{\mid}}
\newcommand{\note}[1]{{\tiny Note: #1}}
\newcommand{\vd}{\vdash}
\newcommand{\of}{\mathrel{:}}
\newcommand{\bind}[2]{#1\mathrel{.} #2}
%\newcommand{\subst}[3]{[#1 \mathbin{/} #2]#3}
\newcommand{\subst}[3]{[#1/#2]#3}

\newcommand{\oversetr}[2]{\overset{#2}{#1}}
\newcommand{\plus}[1]{\overset{+}{#1}}
\newcommand{\minus}[1]{\overset{-}{#1}}

\newcommand{\ace}{\Leftrightarrow} % Algorithmic Constructor Equivalence
\newcommand{\ape}{\leftrightarrow} % Algorithmic Path Equivalence

\newcommand{\Darrow}{\mathrel{\scalebox{1.2}[1]{$\Downarrow$}}}
\newcommand{\whn}{\Darrow}
\newcommand{\whr}{\mathrel{\leadsto}}
\newcommand{\Uarrow}{\mathrel{\scalebox{1.2}[1]{$\Uparrow$}}}
%\newcommand{\nk}{\mathrel{\Uarrow}}
\newcommand{\nk}{\mathrel{\uparrow}}
\newcommand{\lift}{\uparrow}
\newcommand{\sk}{\mathrel{\unlhd}} % TODO: triangle less equal

\newcommand{\synthesis}{\Rightarrow}
\newcommand{\checking}{\Leftarrow}

\newcommand{\comp}{\mathrel{\circ}}

\newcommand{\singleton}[1]{\mathop{S}(#1)}

%\newcommand{\test}[3][]{\ifthenelse{\isempty{#1}}{omitted}{given} #2 #3}
\newcommand{\inferr}[3][]{\inferrule*[Right=#1]{#3}{#2}}


\newcommand{\nats}{\mathcal{N}}
\newcommand{\reals}{\mathcal{R}}

\newcommand{\translatesto}{\mathrel{\leadsto}}

\DeclareMathOperator{\kind}{kind}
\DeclareMathOperator{\ok}{ok}
\DeclareMathOperator{\fv}{FV}
\DeclareMathOperator{\id}{id}
\DeclareMathOperator{\longid}{longid}
%\newcommand{\type}{\texttt{T}
\DeclareMathOperator{\type}{T}
%\DeclareMathOperator{\type}{Type}
\DeclareMathOperator{\refraw}{ref}
\DeclareMathOperator{\intt}{int}
\DeclareMathOperator{\stringt}{string}
\DeclareMathOperator{\unit}{unit}
\DeclareMathOperator{\as}{as}
\newcommand{\env}{\texttt{env}}
\newcommand{\exn}{\texttt{exn}}
\newcommand{\halt}{\texttt{halt}}
\newcommand{\letcc}[3]{\texttt{callcc}_{#1}\bind{#2}{#3}}
\newcommand{\cont}[1]{\texttt{cont}~#1} % TODO
\newcommand{\throw}[3]{\texttt{throw}_{#1} #2 \texttt{ to } #3}
\newcommand{\callcc}{\texttt{callcc}}
\newcommand{\pack}{\texttt{pack}}
\newcommand{\packbind}[3]{\texttt{pack}~\sd{#1}{#2}~\texttt{as}~#3}
\newcommand{\unpackbind}[4]{\texttt{unpack}~[#1, #2] = #3~\texttt{in}~#4~\texttt{end}}
\newcommand{\unpack}{\texttt{unpack}}
%\newcommand{\unpack}[4]{\texttt{unpack}[#1, #2] = #3~\texttt{in}~#4~\texttt{end}}
\newcommand{\sref}[1]{\refraw(#1)}
\newcommand{\letbind}[3]{\texttt{let}~#1 \mathrel{=} #2~\texttt{in}~#3~\texttt{end}}
\newcommand{\letbinds}[2]{\texttt{let}~#1~\texttt{in}~#2~\texttt{end}}
\newcommand{\ap}[2]{#1\mathop{ }#2}
\newcommand{\Ap}[2]{#1\mathop{\cdot}#2}

\newcommand{\pair}[1]{\langle #1 \rangle}
\newcommand{\prodn}[1]{\langle #1_1 \dots #1_n \rangle}
\newcommand{\Prodn}[1]{\times [ #1_1 \dots #1_n ]}

\newcommand{\alloc}[1]{\texttt{alloc}[#1]}
\newcommand{\fresh}[1]{\texttt{fresh}[#1]}
\newcommand{\ext}[1]{\texttt{ext}~#1}
\newcommand{\api}{\Pi^\texttt{ap}}
\newcommand{\alam}[2]{\lambda^\texttt{ap}\bind{#1}{#2}}
\newcommand{\sig}{\texttt{sig}}

%\newcommand{\lheadraw}[1]{
%\begin{tikzpicture}[#1]
%\coordinate (l) at (0, -1.1ex);
%\coordinate (h) at (0, 1.1ex);
%\draw (l) to[out=140, in=220] (h);
%\draw (l) to[out=90, in=270] (h);
%\end{tikzpicture}
%}
%\newcommand{\lhead}{\raisebox{-0.4ex}{\lheadraw{}}}
\newcommand{\lhead}{(\kern-0.2em{\mid}}
\newcommand{\rhead}{)\kern-0.48em{\mid} \kern+0.2em}
%\newcommand{\rhead}{\mid\kern-0.5em{)}}
\newcommand{\lsand}{\langle\kern-0.2em{\mid}}
%\newcommand{\rsand}{\mid\kern-0.5em{\rangle}}
\newcommand{\rsand}{\rangle\kern-0.48em{\mid} \kern+0.2em}
\newcommand{\satom}[1]{\lhead #1\, \rhead}
\newcommand{\datom}[1]{\lsand #1\, \rsand}

\newcommand{\Fst}[1]{\texttt{Fst}(#1)}
\newcommand{\Snd}[1]{\texttt{Snd}(#1)}
\newcommand{\Ext}[1]{\texttt{Ext}\ #1}

\newcommand{\seal}[2]{#1 \mathbin{\texttt{:>}} #2}
\newcommand{\Pig}[2]{\Pi^\texttt{gen}\bind{#1}{#2}}
\newcommand{\lambdag}[2]{\lambda^\texttt{gen}\bind{#1}{#2}}
\newcommand{\Pia}[2]{\Pi^\texttt{app}\bind{#1}{#2}}
\newcommand{\lambdaa}[2]{\lambda^\texttt{app}\bind{#1}{#2}}

\newcommand{\letp}[3]{\texttt{letp}~#1 \mathrel{=} #2~\texttt{in}~#3~\texttt{end}}

\newcommand{\vdp}{\vd_{\kern-0.4em{\textsc{p}}}}
\newcommand{\vdi}{\vd_{\kern-0.4em{\textsc{i}}}}
\newcommand{\vdk}{\vd_{\kern-0.3em{\kappa}}}
%\newcommand{\vdi}{\vd_I}
\newcommand{\splitsto}{\leadsto}
\newcommand{\red}[1]{\textcolor{red}{#1}}
\newcommand{\sd}[2]{\red{[}#1\mathrel{\red{,}}#2\red{]}}
\newcommand{\axiom}[1]{\inferrule{#1}{\strut}}

% TODO ap Ap
\newcommand{\lam}[2]{{\lambda\bind{#1}{#2}}}
\newcommand{\Lam}[2]{{\Lambda\bind{#1}{#2}}}

\newcommand{\name}{\texttt{name}}
\newcommand{\ns}{\texttt{ns}}
\newcommand{\ty}{\texttt{ty}}
\newcommand{\expn}{\texttt{exp}}
\newcommand{\openn}{\texttt{open}}
\newcommand{\typen}{\texttt{type}}
\newcommand{\spec}{\texttt{spec}}
\newcommand{\dec}{\texttt{dec}}
\newcommand{\ec}{\texttt{ec}}
\newcommand{\Gammaec}{\Gamma \vd M_\ec \of \sigma_\ec}
\newcommand{\fn}[2]{\texttt{fn}~#1~\Rightarrow~#2}
\newcommand{\inn}[2]{\texttt{in}_{#1}~#2}
\newcommand{\outn}[1]{\texttt{out}~#1}
\newcommand{\sraise}[1]{\texttt{raise}~#1}
\newcommand{\rtri}{\vartriangleright}
\newcommand{\VAL}[1]{\texttt{VAL}~#1}
\newcommand{\CON}[1]{\texttt{CON}~#1}
\newcommand{\MOD}[1]{\texttt{MOD}~#1}
\newcommand{\FUN}[1]{\texttt{FUN}~#1}
\newcommand{\HIDE}{\texttt{HIDE}}
\newcommand{\structure}[1]{\texttt{structure}~#1}
\newcommand{\struct}[1]{\texttt{struct}~#1~\texttt{end}}
\newcommand{\sign}[1]{\sig~#1~\texttt{end}}
\renewcommand{\mod}{\texttt{mod}}
\newcommand{\local}[2]{\texttt{local}~#1~\texttt{in}~#2~\texttt{end}}
\newcommand{\functor}[2]{\texttt{functor}~#1~(#2)}

\title{HOT Compilation Notes}
\author{Rahul Manne\\
{\tt rmanne}@andrew.cmu.edu}
\date{}

\begin{document}
\maketitle

\section*{Disclaimer/README}
These are only reference notes, and by no means fully capture what
is taught in class.\\

Notes for 170131 (on substitution) are extremely incoherent
so I did not include them by default.\\

There may be errors, feel free to report them to me.\\

\section{Compiler Structure}
SML \\
\tab\trans{elaborate} IL-Module \\
\tab\tab\trans{phase-splitting} IL-Direct \\
\tab\tab\tab\trans{cps conversion} IL-CPS \\
\tab\tab\tab\tab\trans{closure conversion} IL-Closure \\
\tab\tab\tab\tab\tab\trans{hoisting} IL-Hoist \\
\tab\tab\tab\tab\tab\tab\trans{allocation} IL-Alloc \\
\tab\tab\tab\tab\tab\tab\tab\trans{code-generation} C

\include{170124}

\include{170126}

%\include{170131}

\include{170202}

\include{170207}

\include{170209}

\include{170214}

\include{170216}

\include{170221}

\include{170223}

\include{170228}

\include{170302}

\include{170307}

\include{170309}

\include{170321}

\include{170323}

\include{170328}

\include{170330}

\include{170404}

\begin{mathpar}
% NOTE: the \sigma on the bottom is different from the
% \sigma on the top (bottom is subbed by the inhabitant)
% CODING NOTE: we leave out the x in OUR code
% TODO: the below are typing rules
%\inferrule{
%  \Gamma \vd e \of \tau \\
%  \Gamma, x \of \tau \vd_\kappa M \of \sigma
%}{
%  \Gamma_\kappa \vd \letbind{x}{e}{M \of \sigma}
%}
%
%\inferrule{
%  \Gamma, x \vd \Fst{M} \gg c
%}{
%  \Gamma \vd \Fst{\letbind{x}{e}{M}} \gg c
%}

\inferrule{
  \Gamma \vd e \of \tau \splitsto \target{e} \\
  \Gamma, x \of \tau \vd_\kappa M \of \sigma \splitsto \sd{c}{e'}
}{
  \Gamma \vdp \letbind{x}{e}{M \of \sigma} \splitsto
    \sd{c}{\letbind{x}{\target{e}}{e'}}
}

\inferrule{
  \Gamma \vdi M_1 \of \Pia{\alpha \of \sigma_1}{\sigma_2} \splitsto e_1 \\
  \Gamma \vdp M_2 \of \sigma_2 \splitsto \sd{c_2}{e_2} \\
  \Gamma \vd \Fst{M_2} \gg c_2 \\
  \sigma_2 \splitsto \sd{\alpha_2 \of k_2}{\tau_2}
}{
  \Gamma \vdi \Ap{M_1}{M_2} \of \subst{c_2}{\alpha}{\sigma_2} \splitsto
    \unpackbind{\beta}{f}{e_1}{
      \packbind{\ap{\beta}{c_2}}{
        \ap{f[c_2]}{e_2}
      }{\exists\bind{\alpha_2 \of \subst{c_2}{\alpha}{k_2}}{\subst{c_2}{\alpha}{\tau_2}}}
    }
}

\inferrule{
  \Gamma \vdi M_1 \of \sigma_1 \splitsto e_1 \\
  \sigma_1 \splitsto \sd{\alpha_1 \of k_1}{\tau_1} \\
  \Gamma, \alpha/s \of \sigma_1 \vdi M_2 \of \sigma_2 \splitsto e_2 \\
  \sigma_2 \splitsto \sd{\alpha_2 \of k_2}{\tau_2}
}{
  \Gamma \vdi \pair{\alpha/s = M_1, M_2} \of \Sigma\bind{\alpha \of \sigma_1}{\sigma_2}
    \splitsto \\
    \unpackbind{\alpha}{s}{e_1}{
      \unpackbind{\alpha_2}{s_2}{e_2}{
        \packbind{\pair{\alpha, \alpha_2}}{
          \pair{s, s_2}
        }{
          \exists\bind{\beta \of \Sigma\bind{\alpha \of k_1}{k_2}}{
            \subst{\pi_1 \beta}{\alpha_1}{\tau_1}
              \times
            \subst{\pi_1 \beta, \pi_2 \beta}{\alpha, \alpha_2}{\tau_2}
          }
        }
      }
    }
}

\inferrule{
  \Gamma \vdi M \of \sigma \splitsto e
}{
  \Gamma \vdi \seal{M}{\sigma} \of \sigma \splitsto e
}

\inferrule{
  \Gamma \vdi M_1 \of \sigma_1 \splitsto e_1 \\
  \Gamma, \alpha/s \of \vdi M_2 \of \sigma_2 \splitsto e_2
}{
  \Gamma \vdi \letbind{\alpha/s}{M_1}{M_2 \of \sigma_2} \of \sigma_2 \splitsto
    \unpackbind{\alpha}{s}{e_1}{e_2}
}

\inferrule{
  \Gamma \vdp M_1 \of \sigma_2 \splitsto \sd{c_1}{e_1} \\
  \Gamma, \alpha/s \of \sigma_1 \vdi M_2 \of \sigma_2 \splitsto e_2 \\
  \Gamma \vd \Fst{M_1} \gg c_1
}{
  \Gamma \vdi \letp{\alpha/s}{M_1}{M_2} \of \sigma \splitsto
    \letbind{s}{e_1}{\subst{c_1}{\alpha}{e_2}}
}

\inferrule{
  \Gamma \vd e \of \tau \tto \target{e'} \\
  \Gamma, x \of \tau \vdi M \of \sigma \splitsto e'
}{
  \Gamma \vdi \letbind{x}{e}{M} \of \sigma \splitsto \letbind{x}{\target{e}}{e'}
}
\end{mathpar}

% OPTIMIZATION 1: For impure pieces, 4 cases, one where both are pure, one
% where both are impure, and 2 for the mix
%
% OPTIMIZATION 2: Turn things into a tuple, from
% (t1, (t2, ...)) -> \times[t1, t2, ...]
%
% OPTIMIZATION 3: lots of signatures split to just \sd{\alpha \of k}{\unit}
% Don't add that unit to the big tuple at the very end

\begin{mathpar}
  % Extentionality/Retyping rules
  \inferrule{
    \Gamma \vd M \of \satom{k'} \splitsto \sd{c^{\of k'}}{e^{\of \unit}} \\
    \Gamma \vd \Fst{M} \gg c \\
    \Gamma \vd c \of k
  }{
    \Gamma \vd M \of \satom{k} \splitsto \sd{c}{e}
  }

  \inferrule{
    \Gamma \vdp M \of \Pia{\alpha \of \sigma_1}{\sigma_2'} \\
    \Gamma, \alpha/s \of \sigma_1 \vd \Ap{M}{s} \of \sigma_2 \splitsto \sd{c}{e}
  }{
    \Gamma \vdp M \of \Pia{\alpha \of \sigma_1}{\sigma_2} \splitsto
      \sd{\lambda\bind{\alpha \of k_1}{c}}{
        \Lambda\bind{\alpha \of k_1}{\lambda\bind{s \of \subst{\alpha}{\alpha_1}{\tau_1}}{e}}
      }
  }
\end{mathpar}


\include{170411}

\include{170413}

\include{170418}

\include{170425}

\begin{mathpar}
  \inferrule{
    \Gamma \vd M_\ec \of \sigma_\ec \rtri \mod \tto M \of \sigma_1 \\
    \Gamma \vd M_\ec \of \sigma_\ec \rtri \sig \tto \sigma_2 \\
    %\Gamma \vd \sigma_1 \le \sigma_2 % this is wrong, instead:
    %\Gamma \vd M \of \sigma_1 \le \sigma_2 \tto M' : \sigma
    % above is also wrong, because M may not be pure, instead:
    \Gamma, \alpha/s \of \sigma_1 \vd s \of \sigma_1 \le \sigma_2 \tto M' : \sigma
  }{
    \Gamma \vd M_\ec \of \sigma_\ec \rtri \seal{\mod}{\sig} \tto
      \letbind{\alpha/s}{M}{M' \of \sigma_2} \of \sigma_2
      %\seal{M}{\sigma_2} \of \sigma_2
  }

  % NOTE: this is basically the same as the above, except
  % ``transparent description'' rather than ``opaque description'' (above)
  \inferrule{
    \Gamma \vd M_\ec \of \sigma_\ec \rtri \mod \tto M \of \sigma_1 \\
    \Gamma \vd M_\ec \of \sigma_\ec \rtri \sig \tto \sigma_2 \\
    \Gamma, \alpha/s \of (\HIDE \of \sigma_1) \vd \outn{s}
      \of \singleton{\alpha \of \sigma_1} \le \sigma_2 \tto M' : \sigma
  }{
    %\Gammaec \rtri \mod \of \sig \tto \letbind{\alpha/s}{M}{M' \of \sigma} \of \sigma
    % above is wrong because avoidance problem
    \Gammaec \rtri \mod \of \sig \tto \pair{\inn{\HIDE}{M}, M'}
      \of \Sigma\bind{\alpha \of (\HIDE \of \sigma_1)}{\sigma}
  }
\end{mathpar}

\subsection{$\plus\Gamma \vd \plus{M} \of \plus\sigma_1 \le \plus\sigma_2 \tto \minus{M'} \of \minus\sigma$}
\begin{mathpar}
  \inferrule{
    \Gamma \vd M \of \sigma \le \sigma_1 \tto M_1 \of \sigma_1' \\
    \Gamma, \alpha \of \Fst{\sigma_1'} \vd M \of \sigma \le \sigma_2 \tto M_2 \of \sigma_2'
    % CODING NOTE: DONT FORGET TO LIFT M and \sigma, since htey come under
    % a binding
  }{
    \Gamma \vd M \of \sigma \le \Sigma\bind{\alpha \of \sigma_1}{\sigma_2} \tto
      \pair{\alpha/\_ = M_1, M_2} \of \Sigma\bind{\alpha \of \sigma_1'}{\sigma_2'}
  }

  \inferrule{
  }{
    \Gamma \vd M \of \sigma \le 1 \tto \ast \of 1
  }

  \inferrule{
    \Gamma \vd M \of \sigma \rtri \VAL{\id} \tto M' \of \datom{\tau'} \\
    \Gamma \vd \tau \equiv \tau' \of \type
  }{
    \Gamma \vd M \of \sigma \le (\VAL{\id} \of \datom{\tau}) \of \sigma' \tto
      \inn{\VAL{\id}}{M'} \of (\VAL{\id} \of \datom{\tau'})
      % for the last piece, \tau and \tau' are equally correct
  }

  \inferrule{
    \Gamma \vd M \of \sigma \rtri \CON{\id} \tto M' \of \satom{k'} \\
    \Gamma \vd k \equiv k' \of \type
  }{
    \Gamma \vd M \of \sigma \le (\CON{\id} \of \satom{k}) \of \sigma' \tto
      \inn{\CON{\id}}{M'} \of (\CON{\id} \of \satom{k'})
  }

  \inferrule{
    \Gamma \vd M_1 \of \sigma_1 \rtri \MOD{\id} \tto M_2 \of \sigma_2 \\
    \Gamma \vd M_2 \of \sigma_2 \le \sigma \tto M_3 \of \sigma_3
  }{
    \Gamma \vd M_1 \of \sigma_1 \le \MOD{\id} \of \sigma \tto
      \inn{\MOD{\id}}{M_3} \of (\MOD{\id} \of \sigma_3)
  }
\end{mathpar}

% TODO: back ot functors
\subsection{}
\begin{mathpar}
  \inferrule{
    \Gamma, \alpha_1 \of \type, \dots, \alpha_n \of \type \vd
      \pair{\pair{M_\ec, \inn{\CON{\id_1}}{\alpha_1}}, \dots, \inn{\CON{\id_n}}{\alpha_n}} \\
      \of \sigma_\ec \times (\CON{\id_1} \of \satom{\singleton{\alpha_1}}) \times \dots \times
                            (\CON{\id_n} \of \satom{\singleton{\alpha_n}})
      \rtri \ty \tto \tau
  }{
    \Gammaec \rtri_\dec \typen(\id_1, \dots, \id_n) \id = \ty \tto \\
      \inn{\CON{\id}}{\satom{\lambda\bind{\alpha_1 \of \type}{\dots \lambda\bind{\alpha_n \of \type}{\tau}}}}
      %\of (\CON{\id} \of \satom{\type \arrow \dots \arrow \type \arrow \type})
      % NOTE above is wrong, below is right
      \of  (\CON{\id} \of \satom{\Pi\bind{\alpha_1 \of \type}{\dots \Pi\bind{\alpha_n \of \type}{\singleton{\tau}}}})
      % total of n + 1 \types
  }

  % basically same as above
  \inferrule{
    \Gamma, \alpha_1 \of \type, \dots, \alpha_n \of \type \vd
      \pair{\pair{M_\ec, \inn{\CON{\id_1}}{\alpha_1}}, \dots, \inn{\CON{\id_n}}{\alpha_n}} \\
      \of \sigma_\ec \times (\CON{\id_1} \of \satom{\singleton{\alpha_1}}) \times \dots \times
                            (\CON{\id_n} \of \satom{\singleton{\alpha_n}})
      \rtri \ty \tto \tau
  }{
    \Gammaec \rtri_\spec \typen(\id_1, \dots, \id_n) \id = \ty \tto
      (\CON{\id} \of \satom{\Pi\bind{\alpha_1 \of \type}{\dots \Pi\bind{\alpha_n \of \type}{\singleton{\tau}}}})
  }

  \inferrule{
  }{
    \Gammaec \rtri_\spec \typen(\id_1, \dots, \id_n) \id = \ty \tto
      (\CON{\id} \of \satom{\type \arrow \dots \arrow \type \arrow \type})
      % total of n + 1 \types
  }

  \inferrule{
    \Gammaec \rtri \ty_i \tto \tau_i \\
    \Gammaec \rtri \longid @ \CON{} \tto M \of \satom{k} \\
    \Gamma \vd k \le \type \arrow \dots \type \arrow \type \\ % total of n + 1 \types
    \Gamma \vd \Fst{M} \gg c
  }{
    \Gammaec \rtri_\spec (\ty_1, \dots, \ty_n) \longid \tto
      \ap{\ap{c}{\tau_1}}{\dots \tau_2}
  }

  \inferrule{
    \Gammaec \rtri \dec_1 \tto M_1 \of \sigma_1 \\
    \Gamma, \alpha/s \of (\HIDE~\sigma_1) \vd \pair{M_\ec, \outn{s}}
      \of \sigma_\ec \times \sigma_1 \rtri \dec_2 \tto M_2 \of \sigma_2
  }{
    \Gammaec \rtri \local{\dec_1}{\dec_2} \tto
      \pair{\alpha/s = \HIDE~M_1, M_2} \of \Sigma\bind{\alpha \of (\HIDE~\sigma_1)}{\sigma_2}
  }

  \inferrule{
    \Gammaec \rtri \sign{\spec_1, \dots, \spec_n} \tto \sigma_1 \\
    \Gamma, \alpha/s \of \sigma_1 \vd \pair{M_\ec, s}
      \of \sigma_\ec \times \singleton{\alpha \of \sigma_1} \rtri \mod \tto M \of \sigma_2
  }{
    \Gammaec \rtri \functor{\id}{\spec_1, \dots, \spec_n} = \mod \tto
      \inn{\FUN{\id}}{\lambdag{\alpha/s \of \sigma_1}{M}}
      \of (\FUN{\id} \of \Pig{\alpha \of \sigma_1}{\sigma_2})
  }
\end{mathpar}

\vspace{1cm}
Now lets look at functor application
\begin{mathpar}
  \inferrule{
    \Gammaec \rtri \longid @ \FUN{} \tto F \of \Pig{\alpha \of \sigma_1}{\sigma_2} \\
    \Gammaec \rtri \dec \tto M \of \sigma \\
    \Gamma, \beta/s \of (\HIDE~\sigma) \vd \outn{s}
      \of \singleton{\beta \of \sigma} \le \sigma_1
      \tto M' \of \sigma_1' \\
    %\Gamma, \beta/s \of (\HIDE \of \sigma) \vd \Fst{M'} \gg c % still has problems
  }{
    \Gammaec \rtri \longid~\dec \tto
      \pair{\beta/s = \inn{\HIDE}{M}, \pair{\gamma/s' = \inn{\HIDE}{M'}, \ap{F}{s'}}} \\
        \of \Sigma\bind{\beta \of (\HIDE \of \sigma)}
                       {\Sigma\bind{\gamma \of (\HIDE \of \sigma_1')}
                                   {\subst{\gamma}{\alpha}{\sigma_2}}}
      %\letbind{\beta/s}{M}{\ap{F}{M'} \of \sigma_2} % this runs into avoidance problem
  }
\end{mathpar}


\end{document}
